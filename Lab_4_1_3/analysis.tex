\section{Обработка результатов измерений}
%\subsection{Обработка результатов}
Найдем показатели преломления воды, глицерина, спирта и двух стеклышек:
% Table generated by Excel2LaTeX from sheet 'Лист1'
\begin{table}[htbp]
	\centering
	\caption{Показатели преломления}
	\begin{tabular}{|l|c|c|c|c|c|}
		\hline
		& \multicolumn{1}{l|}{Вода} & \multicolumn{1}{l|}{Спирт} & \multicolumn{1}{l|}{Глицерин} & \multicolumn{1}{l|}{Стекло 1}& \multicolumn{1}{l|}{Стекло 2}\\
		\hline
		Ск. луч & $1.3319\pm0.0006$ & $1.3604\pm0.0006$ & $1.4685\pm0.0007$  & 1.5140&	1.6165
		\\
		\hline
		Полн. отр & 1.3320 & 1.3605 & 1.4695 & -&-\\
		\hline
	\end{tabular}%
	\label{tab:addlabel}%
\end{table}%

 % Table generated by Excel2LaTeX from sheet 'Лист1'
\begin{table}[htbp]
	\centering
	\caption{Табличные показатели преломления}
	\begin{tabular}{|c|c|}
		\hline
		& Таблич \bigstrut\\
		\hline
		Вода & \multicolumn{1}{c|}{1.333} \bigstrut\\
		\hline
		Спирт & \multicolumn{1}{c|}{1.362} \bigstrut\\
		\hline
		Глицерин & \multicolumn{1}{c|}{1.470} \bigstrut\\
		\hline
		Стекло & 1.46--2.04 \bigstrut\\
		\hline
	\end{tabular}%
	\label{tab:addlabel}%
\end{table}%



Вычислим молекулярные рефракции  и поляризуемость (в СГС) воды, спирта и глицерина по формулам:
\begin{equation}
R_M = \frac{M}{\rho}\,\frac{n^2-1}{n^2+2} = \frac{4\pi}{3}\,N_a\alpha
\end{equation}
Получили значения:
% Table generated by Excel2LaTeX from sheet 'Лист1'
\begin{table}[htbp]
	\centering
	\caption{Метод скользящего луча}
	\begin{tabular}{|l|c|c|}
		\hline
		 & \multicolumn{1}{l|}{$R_M$, СГС} & \multicolumn{1}{l|}{$\alpha$, СГС} \\
		\hline
		Вода  & 3.69  & 1.46E-24 \\
		\hline
		Глицерин & 20.32 & 8.06E-24 \\
		\hline
		Спирт & 12.88 & 5.11E-24 \\
		\hline
	\end{tabular}%
	\label{tab:addlabel}%
\end{table}%

% Table generated by Excel2LaTeX from sheet 'Лист1'
\begin{table}[h!]
	\centering
	\caption{Метод полного внутреннего отражения}
	\begin{tabular}{|l|c|c|}
		\hline
		 & \multicolumn{1}{l|}{$R_M$, СГС} & \multicolumn{1}{l|}{$\alpha$, СГС} \\
		\hline
		Вода  & 3.69 & 1.46E-24 \\
		\hline
		Глицерин & 20.35 & 8.07E-24 \\
		\hline
		Спирт & 12.88 & 5.11E-24 \\
		\hline
	\end{tabular}%
	\label{tab:addlabel}%
\end{table}%

С помощью системы:
\begin{equation}
\begin{pmatrix}
0 & 2 & 1\\ 3 & 8 & 3\\2&6&1\\
\end{pmatrix}
\begin{pmatrix}
R_C\\R_H\\R_O
\end{pmatrix}
=
\begin{pmatrix}
R_{H_2O}\\
R_{C_3H_8O_3}\\
R_{C_2H_6O}\\
\end{pmatrix}
\end{equation}
найдем атомные рефракции углерода, водорода и кислорода:
% Table generated by Excel2LaTeX from sheet 'Лист1'
\begin{table}[h!]
	\centering
	\caption{Атомные рефракции}
	\begin{tabular}{|l|c|c|}
		\hline
		& \multicolumn{1}{l|}{Ск. луч} & \multicolumn{1}{l|}{Полн. отр} \\
		\hline
		$R_C$  & 2.32 & 2.34 \\
		\hline
		$R_H$  & 1.14 & 1.13 \\
		\hline
		$R_O$  & 1.42 & 1.44 \\
		\hline
	\end{tabular}%
	\label{tab:addlabel}%
\end{table}%

Полагая справедливым правило аддитивности, вычислим молекулярную рефракцию метилового спирта и показатели преломления метилового спирта, льда и алмаза:
 % Table generated by Excel2LaTeX from sheet 'Лист1'
\begin{table}[h!]
	\centering
	\caption{Молекулярная дифракция мет. спирта и показатели преломления}
	\begin{tabular}{|l|c|c|c|}
		\hline
		& \multicolumn{1}{l|}{Ск. луч} & \multicolumn{1}{l|}{Полн. отр} & \multicolumn{1}{l|}{Таблич} \bigstrut\\
		\hline
		$R_{CH_4O}$ & 8.29 & 8.29 & \multicolumn{1}{c|}{-} \bigstrut\\
		\hline
		$n_{CH_4O}$ & 1.3319 & 1.3320& 1.331 \bigstrut\\
		\hline
		$n_{\text{лед}}$ & 1.3018 & 1.3018 & 1.31 \bigstrut\\
		\hline
		$n_{\text{алмаз}}$ & 2.7030 & 2.7296  & 2.42 \bigstrut\\
		\hline
	\end{tabular}%
	\label{tab:addlabel}%
\end{table}%





